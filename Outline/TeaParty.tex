\documentclass[12pt]{article}
\usepackage[utf8]{inputenc}
\usepackage{booktabs}
\usepackage{multirow}
\usepackage{rotating}
\usepackage{bigstrut}

%% fonts
\usepackage[utopia]{mathdesign}
\usepackage[scaled=.95]{inconsolata}

%% page margins, inter-paragraph space and no chapters
\usepackage[margin=1.1in]{geometry}
\setlength{\parskip}{0.5em}
\renewcommand{\thesection}{\arabic{section}}

%% For actual bib
\usepackage{natbib}
\bibpunct{(}{)}{,}{author-year}{}{,}

%% Fuck with the title
\makeatletter
\renewcommand{\maketitle}{\bgroup\setlength{\parindent}{0pt}
\begin{flushleft}
  \textbf{\@title}

  \@author
\end{flushleft}\egroup
}
\makeatother

%% for memisc
\usepackage{booktabs}
\usepackage{dcolumn}

%% define a dark blue
\usepackage{color}
\definecolor{darkblue}{rgb}{0,0,.5}

%% hyperlinks to references
\usepackage{hyperref}
\hypersetup{colorlinks=true, linkcolor=darkblue, citecolor=darkblue, filecolor=darkblue, urlcolor=darkblue}

\author{Adam Olson\\University of Minnesota\\February 16, 2014}
\title{The Tea Party Does Not Exist}
\date{February 16, 2014}

\begin{document}
\maketitle

The historical trajectory of conservatism in Congress 

Since the start of the 20th century, the ideological trajectory of the Republican Party in the United States Congress has ebbed and flowed between the moderate and conservative wings of the party. Over the last 50 years, however, the Republican Party has steadily moved to the right ideologically.


If there has not been significant changes in voting patterns or pivot points, then why do some Republicans adopt the tea party mantle? They do so to aid their election. The overwhelming majority of TP members are quality challengers and there is a huge new pool of money for Republicans to use should they
call themselves TPers. Additionally, the endorsement of these TP groups plays a powerful cue to Republican primary voters that the candidate is sufficiently conservative for that style of voters.

\section{The Rise of Conservatism in the Republican Party}

\section{Primary Elections and Conservative Republicans}

\section{Ideological Changes in Congressional Republicans}
\subsection{Macro Level Changes}
The power of very conservative members in the Congressional Republican Party was not created with the emergence of the Tea Party. Instead, Congressional Republicans have been moving farther and farther right for the last 30 years. In this respect, the Tea Party is just a new label for the most conservative Republicans.

\subsection{Micro Level Changes}
On an individual level, the Congressional TP members are not significantly more conservative than non TP members in previous congresses. Moreover, many of the MCs who are claiming TP identification now are just the very conservative members previously.

\section{The Tea Party Electoral Connection Revisited}
Conservative MCs adopt the TP moniker for electoral benefits.

\newpage
    \bibliography{TeaParty}{}
\bibliographystyle{jpe}

%\printbibliography
\end{document}


