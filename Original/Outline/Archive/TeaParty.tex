\documentclass[12pt]{article}
\usepackage[utf8]{inputenc}
\usepackage{booktabs}
\usepackage{multirow}
\usepackage{rotating}
\usepackage{bigstrut}

%% fonts
\usepackage[utopia]{mathdesign}
\usepackage[scaled=.95]{inconsolata}

%% page margins, inter-paragraph space and no chapters
\usepackage[margin=1.1in]{geometry}
\setlength{\parskip}{0.5em}
\renewcommand{\thesection}{\arabic{section}}

%% For actual bib
\usepackage{natbib}
\bibpunct{(}{)}{,}{author-year}{}{,}

%% Mess with the title
\makeatletter
\renewcommand{\maketitle}{\bgroup\setlength{\parindent}{0pt}
\begin{flushleft}
  \textbf{\@title}

  \@author
\end{flushleft}\egroup
}
\makeatother

%% for memisc
\usepackage{booktabs}
\usepackage{dcolumn}

%% define a dark blue
\usepackage{color}
\definecolor{darkblue}{rgb}{0,0,.5}

%% hyperlinks to references
\usepackage{hyperref}
\hypersetup{colorlinks=true, linkcolor=darkblue, citecolor=darkblue, filecolor=darkblue, urlcolor=darkblue}

\author{Adam Olson\\University of Minnesota\\February 16, 2014}
\title{The Tea Party Does Not Exist}
\date{February 26, 2014}

\begin{document}
\maketitle

Over the last five years, the conventional wisdom that the rise of the Tea Party is evidence of a fundamental shift in the Republican Party, if not America overall. Political commentators in American popular culture spend much of their time trying to determine just how much influence this errant faction of the Republican Party has on public policy and the political process. (Insert a bunch of stuff here)

This paper takes a counter stance and questions the existance of the Tea Party in Congress. 

Instead, the Tea Party is not a new ideological phenomenon, it is just the name that 

Conservatives have shifted from opposing government action en masse to supporting it when it aligns with their preferences.

Most of the literatrue surrounding the Tea Party has been based on the way everyday citizens have or have not embraced the TP movement (See Skocpol book). These sorts of papers have tried to identify predictors of TP support (cite), contextual factors within which TP support thrives (cite), and the ways how media outlets frame the TP (cite). At the same time, there has been very little attention paid in the scholarly literature to the effects of the TP  


The historical trajectory of conservatism in Congress 

Since the start of the 20th century, the ideological trajectory of the Republican Party in the United States Congress has ebbed and flowed between the moderate and conservative wings of the party. Over the last 50 years, however, the Republican Party has steadily moved to the right ideologically.




If there has not been significant changes in voting patterns or pivot points, then why do some Republicans adopt the tea party mantle? They do so to aid their election. The overwhelming majority of TP members are quality challengers and there is a huge new pool of money for Republicans to use should they
call themselves TPers. Additionally, the endorsement of these TP groups plays a powerful cue to Republican primary voters that the candidate is sufficiently conservative for that style of voters.

Most of the work on the Tea Party

\section{The Rise of Activist Conservativism}


\section{Primary Elections and Conservative Republicans}

\section{Ideological Changes in Congressional Republicans}
\subsection{Macro Level Changes}
The power of very conservative members in the Congressional Republican Party was not created with the emergence of the Tea Party. Instead, Congressional Republicans have been moving farther and farther right for the last 30 years. In this respect, the Tea Party is just a new label for the most conservative Republicans.

\subsection{Micro Level Changes}
On an individual level, the Congressional TP members are not significantly more conservative than non TP members in previous congresses. Moreover, many of the MCs who are claiming TP identification now are just the very conservative members previously.

\section{Agenda Setting, Pivot Points and Roll Rates}

\subsection{Roll Rates in the 112th Congress}
One useful component about the 112th Congress being the first Congress with a ``tea party'' preference is that the Republican Party was the majority party in the House of Representatives.
When a party has majority party status in the House, they have the the ability to control the agenda both positively and negatively (CITE).\footnote{While I am relying on the partisan theories to discuss agenda control, the use of preference based theories such as Krehbiel would not fundamentally change this argument. If the Republican majority is often rolled then we have evidence of a tea party effect.} Put differently, the party will be able to bring up bills it wants to discuss (positive agenda control) stop bills it does not want to discuss (negative agenda control). and  One aspect of this control is that the majority party tries to ensure that it will not be rolled. The party is rolled when it loses a vote. Much research argues that the majority party would not allow a vote on a bill that they would lose and also the majority party would usually win votes on bills put up for a vote. The logic is that the majority party controls the agenda and they simply would not try to advance legislation that fits in either of these categories.

By examining roll rates in the 112th Congress we can see how often the Republican majority wound up on the losing side of a vote. Previous literature suggests that 

%I don't know about the individual roll rate paragraph. What does it really tell us? We're looking for majority getting screwed. Who cares if individuals are on the losing side of things?
We can also use individual roll rates -- that is how often the member is on the losing side of a vote -- to find a tea party effect. Presumably 

Another way we can look at this is to examine roll rates as they concern special rule adoption. Special Rules are one of the main ways the Majority party shepherds l

\section{The Tea Party Electoral Connection Revisited}
Conservative MCs adopt the TP moniker for electoral benefits.

\newpage
    \bibliography{TeaParty}{}
\bibliographystyle{jpe}

%\printbibliography
\end{document}


